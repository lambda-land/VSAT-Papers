% Created 2020-06-04 Thu 15:31
% Intended LaTeX compiler: pdflatex
\documentclass[presentation]{beamer}
\usepackage[utf8]{inputenc}
\usepackage[T1]{fontenc}
\usepackage{graphicx}
\usepackage{grffile}
\usepackage{longtable}
\usepackage{wrapfig}
\usepackage{rotating}
\usepackage[normalem]{ulem}
\usepackage{amsmath}
\usepackage{textcomp}
\usepackage{amssymb}
\usepackage{capt-of}
\usepackage{hyperref}
\usepackage{minted}
\AtBeginSection[]{\begin{frame}<beamer>\frametitle{Topic}\tableofcontents[currentsection]\end{frame}}
\usepackage{tikz}
\usepackage{times}
\usepackage{amsmath}
\usepackage{verbatim}
\usetikzlibrary{shapes.geometric,backgrounds, positioning-plus,node-families,calc}
\usetheme{boxes}
\author{Jeffrey Young}
\date{2020-05-31}
\title{VSAT: What is Next}
\hypersetup{
 pdfauthor={Jeffrey Young},
 pdftitle={VSAT: What is Next},
 pdfkeywords={},
 pdfsubject={},
 pdfcreator={Emacs 26.3 (Org mode 9.3.6)},
 pdflang={English}}
\begin{document}

\maketitle
\begin{frame}{Outline}
\tableofcontents
\end{frame}


\section{Review: The Algorithm}
\label{sec:org388f87a}

\begin{frame}[label={sec:orgccc328b}]{The Algorithm: Architecture}
% layers
\pgfdeclarelayer{background}
\pgfdeclarelayer{foreground}
\pgfdeclarelayer{background,main,foreground}

% styles
\tikzstyle{nd}=[rectangle, draw, fill=white, text width=7em,text centered, minimum height=2.5em, scale=0.8]
\tikzstyle{ann}=[above, text width=5em]
\tikzstyle{lbl} = [scale=0.8,text width=7em, text centered]
 \def\blockdist{2.3}
 \def\edgedist{2.5}

 \begin{picture}(320,200)
 % \put(-14, 5){
 \put(-14, -10){
 \begin{tikzpicture}[node distance=4.5cm,>=latex']
  %% Nodes
  \node (toIL)  [nd] {Parse to IL};
  \node (AccEv) [nd, right of=toIR]  {Evaluation \\ Accumulation};
  \node (input) [lbl, above=0.5cm of toIL] {vpl};

  \node (solve) [nd, right of=AccEv] {Solve VCore};
  \node (VModel) [nd, below=2.5cm of solve] {VModel \\ Constructor};
  \node (output) [lbl, below=0.5cm of VModel] {Variational \\ Model};

  %% Arrows
  \draw [draw, ->] (input) -- node {} (toIL);
  \draw [draw, ->] (toIL) -- node {} (AccEv);
  \draw [draw, ->] (AccEv) -- node {} (solve);
  \path[]
    (solve) edge [->,bend right=45] node {} (VModel)
    (solve) edge [->,bend right=35] node {} (VModel)
    (solve) edge [->,bend right=25] node {} (VModel)

    (solve) edge [->,bend left=25] node {} (VModel)
    (solve) edge [->,bend left=35] node {} (VModel)
    (solve) edge [->,bend left=45] node {} (VModel);
  \draw [draw, ->] (VModel) -- node {} (output);
\end{tikzpicture}
}
\end{picture}
\begin{block}{\phantom{m}}
\end{block}
\end{frame}

\begin{frame}[label={sec:org05038aa}]{The Algorithm: But Really}
% layers
\pgfdeclarelayer{background}
\pgfdeclarelayer{foreground}
\pgfdeclarelayer{background,main,foreground}

% styles
% \tikzstyle{nd}=[rectangle, fill=blue!20, text width=7em,text centered, minimum height=2.5em]
\tikzstyle{nd}=[rectangle, draw, fill=white, text width=7em,text centered, minimum height=2.5em, scale=0.8]
\tikzstyle{ann}=[above, text width=5em]
\tikzstyle{lbl} = [scale=0.8,text width=7em, text centered]
 \def\blockdist{2.3}
 \def\edgedist{2.5}

 \begin{picture}(320,200)
 \put(-25, -10){
 \begin{tikzpicture}[node distance=4.5cm,>=latex']
  %% Nodes
  \node (toIL)  [nd] {Parse to IL};
  \node (AccEv) [nd, right of=toIR]  {Evaluation \\ Accumulation};
  \node (input) [lbl, above=0.5cm of toIL] {vpl};
  \node (pls)   [lbl, above right=0.25cm of input] {$\{pl_{0}, pl_{1}, \ldots pl_{n}\}$};
  %%\node (VModel) [nd] at (7.2,-2) {VModel \\ Constructor};

  \node (solve) [nd, right of=AccEv] {Solve VCore};
  \node (VModel) [nd, below=2.5cm of solve] {VModel \\ Constructor};
  \node (output) [lbl, below=0.5cm of VModel] {Variational \\ Model};

  %% labels
  \node (SymExlbl) [lbl, below right=0.3cm and -0.5cm of toIL] {Symbolic \\ Execution};
  \node (Solvelbl) [lbl, right=-0.3cm of solve] {Actual \\ Execution};
  \node (sbvLbl)   [lbl, below=0.45cm of SymExlbl] {SBV};

  %% Arrows
  \draw [draw, ->] (input) -- node {} (toIL);
  \draw [draw, ->] (toIL) -- node {} (AccEv);
  \draw [draw, ->] (AccEv) -- node {} (solve);
  \path[]
    (solve) edge [->,bend right=45] node {} (VModel)
    (solve) edge [->,bend right=35] node {} (VModel)
    (solve) edge [->,bend right=25] node {} (VModel)

    (solve) edge [->,bend left=25] node {} (VModel)
    (solve) edge [->,bend left=35] node {} (VModel)
    (solve) edge [->,bend left=45] node {} (VModel);
  \draw [draw, ->] (VModel) -- node {} (output);
  \path[]
    (pls) edge [->,bend right=45] node {} (input)
    (pls) edge [->,bend right=35] node {} (input)
    (pls) edge [->,bend right=25] node {} (input);


 %% Background
 \begin{pgfonlayer}{background}
   % sym ex
   \path (toIL.west |- AccEv.north)+(-0.25,0.3) node (a) {};
   \path (SymExlbl.south -| AccEv.east)+(0.25,0.0) node (b) {};

   % act ex
   \path (solve.north west)+(-0.125,0.15) node (c) {};
   \path (Solvelbl.south east)+(-0.3,-0.15) node (d) {};

   % sbv
   \path (sbvLbl.south -| Solvelbl. sourth east)+(-0.3,-0.15) node (e) {};

   % boxes
   \path[fill=blue!20,rounded corners, draw=black!50, dashed] (a) rectangle (e);
   \path[fill=yellow!20,rounded corners, draw=black!50, dashed] (a) rectangle (b);
   \path[fill=red!20,rounded corners, draw=black!50, dashed] (c) rectangle (d);

   % \node[background, fill=blue!20, draw=black!20, dashed, rounded corners
   % , fit={(frontend) (toIL) (AccEV) ($(solve.west)+(3.0,-1.5)$)}, label=below:sbv] {};
   % \node[background, fill=yellow!20, draw=black!50, dashed, rounded corners, fit=(toIL) (AccEv), label=below:\nodeSymbolic Execution] {};
   % \node[background, fill=red!20, draw=black!50, dashed, rounded corners, fit=(solve) (solve), label=right:Actual \\ Execution] {};

 \end{pgfonlayer}

\end{tikzpicture}
}
\end{picture}

\begin{block}{\phantom{m}}
\end{block}
\end{frame}
\begin{frame}[label={sec:org3386365}]{The Algorithm: Which means}
% layers
\pgfdeclarelayer{background}
\pgfdeclarelayer{foreground}
\pgfdeclarelayer{background,main,foreground}

% styles
% \tikzstyle{nd}=[rectangle, fill=blue!20, text width=7em,text centered, minimum height=2.5em]
\tikzstyle{nd}=[rectangle, draw, fill=white, text width=7em,text centered, minimum height=2.5em, scale=0.8]
\tikzstyle{ann}=[above, text width=5em]
\tikzstyle{lbl} = [scale=0.8,text width=7em, text centered]
 \def\blockdist{2.3}
 \def\edgedist{2.5}

 \begin{picture}(320,200)
 \put(-25, -10){
 \begin{tikzpicture}[node distance=4.5cm,>=latex']
  %% Nodes
  \node (toIL)  [nd] {Parse to IL};
  \node (AccEv) [nd, right of=toIR]  {Evaluation \\ Accumulation};
  \node (input) [lbl, above=0.5cm of toIL] {vpl};
  \node (pls)   [lbl, above right=0.25cm of input] {$\{pl_{0}, pl_{1}, \ldots pl_{n}\}$};
  %%\node (VModel) [nd] at (7.2,-2) {VModel \\ Constructor};

  \node (solve) [nd, right of=AccEv] {Solve VCore};
  \node (VModel) [nd, below=2.5cm of solve] {VModel \\ Constructor};
  \node (output) [lbl, below=0.5cm of VModel] {Variational \\ Model};

  %% labels
  \node (SymExlbl) [lbl, below right=0.3cm and -0.5cm of toIL] {Symbolic \\ Execution};
  \node (Solvelbl) [lbl, right=-0.3cm of solve] {Actual \\ Execution};
  \node (sbvLbl)   [lbl, below=0.45cm of SymExlbl] {SBV};

  %% Arrows
  \draw [draw, ->] (input) -- node {} (toIL);
  \draw [draw, ->] (toIL) -- node {} (AccEv);
  \draw [draw, ->] (AccEv) -- node {} (solve);
  \path[]
    (solve) edge [->,bend right=45] node {} (VModel)
    (solve) edge [->,bend right=35] node {} (VModel)
    (solve) edge [->,bend right=25] node {} (VModel)

    (solve) edge [->,bend left=25] node {} (VModel)
    (solve) edge [->,bend left=35] node {} (VModel)
    (solve) edge [->,bend left=45] node {} (VModel);
  \draw [draw, ->] (VModel) -- node {} (output);
  \path[]
    (pls) edge [->,bend right=45] node {} (input)
    (pls) edge [->,bend right=35] node {} (input)
    (pls) edge [->,bend right=25] node {} (input);


 %% Background
 \begin{pgfonlayer}{background}
   % sym ex
   \path (toIL.west |- AccEv.north)+(-0.25,0.3) node (a) {};
   \path (SymExlbl.south -| AccEv.east)+(0.25,0.0) node (b) {};

   % act ex
   \path (solve.north west)+(-0.125,0.15) node (c) {};
   \path (Solvelbl.south east)+(-0.3,-0.15) node (d) {};

   % sbv
   \path (sbvLbl.south -| Solvelbl. sourth east)+(-0.3,-0.15) node (e) {};

   % boxes
   \path[fill=blue!20,rounded corners, draw=black!50, dashed] (a) rectangle (e);
   \path[fill=yellow!20,rounded corners, draw=black!50, dashed] (a) rectangle (b);
   \path[fill=red!20,rounded corners, draw=black!50, dashed] (c) rectangle (d);

   % \node[background, fill=blue!20, draw=black!20, dashed, rounded corners
   % , fit={(frontend) (toIL) (AccEV) ($(solve.west)+(3.0,-1.5)$)}, label=below:sbv] {};
   % \node[background, fill=yellow!20, draw=black!50, dashed, rounded corners, fit=(toIL) (AccEv), label=below:\nodeSymbolic Execution] {};
   % \node[background, fill=red!20, draw=black!50, dashed, rounded corners, fit=(solve) (solve), label=right:Actual \\ Execution] {};

 \end{pgfonlayer}

\end{tikzpicture}
}
\end{picture}

\begin{block}{Coupling!}
\begin{itemize}
\item sbv responsible for \textasciitilde{}90\% of runtime performance
\end{itemize}
\end{block}
\end{frame}

\section{Essential steps}
\label{sec:org8d1420a}

\begin{frame}[label={sec:org6bb5b89}]{Input}
% layers
\pgfdeclarelayer{background}
\pgfdeclarelayer{foreground}
\pgfdeclarelayer{background,main,foreground}

% styles
% \tikzstyle{nd}=[rectangle, fill=blue!20, text width=7em,text centered, minimum height=2.5em]
\tikzstyle{nd}=[rectangle, draw, fill=white, text width=7em,text centered, minimum height=2.5em, scale=0.8]
\tikzstyle{ann}=[above, text width=5em]
\tikzstyle{lbl} = [scale=0.8,text width=7em, text centered]
 \def\blockdist{2.3}
 \def\edgedist{2.5}

 \begin{picture}(320,100)
 \put(0, 0){
 \begin{tikzpicture}
 \node (pls) {%
 $\begin{aligned}
 \color{blue}c_{0.0} &\wedge c_{0.1} \wedge c_{0.2} \ldots c_{0.n}\\
 c_{1.0} &\wedge \textcolor{red}{c_{1.1}} \wedge c_{1.2} \ldots c_{1.n}\\
 \vdots\\
 c_{m.0} &\wedge \textcolor{blue}{c_{m.1}} \wedge c_{m.2} \ldots c_{m.n}
 \end{aligned}$};
% \begin{tikzpicture}[node distance=4.5cm,>=latex']
%  %% Nodes
  % \node (input) [lbl] at (8,-10) {vpl};
  \node (input) [lbl, right=3cm of pls] {vpl};
  %% Arrows
  \path[]
    (pls) edge [->,bend right=45] node {} (input)
    (pls) edge [->,bend right=35] node {} (input)
    (pls) edge [->,bend right=25] node {} (input);
\end{tikzpicture}
}
\end{picture}

\begin{block}{Way to identify shared vs non-shared terms}
\begin{itemize}
\item Transform set of plain formulas to a vpl formula
\end{itemize}
\end{block}

\begin{block}{Possible performance gains}
\begin{itemize}
\item How to find quality encoding, Relation to BDD work?
\item High impact for other areas of applied choice calculus
\end{itemize}
\end{block}
\end{frame}

\begin{frame}[label={sec:org39fe254}]{Variational Core}
\begin{picture}(320,250)
\put(6,116){\begin{tikzpicture}
\Tree [.$\vee$ [.$\vee$ [.$\chc[A]{}$ ] [. $s_{0}$ ] ]
      [.$\wedge$ [.$\chc[D]{}$ ][.$\wedge$ [. $\chc[B]{}$ ][ .$s_{1}$ ] ] ] ]
\end{tikzpicture}}
\put(150,220){\begin{minipage}[t]{0.6\linewidth}
\begin{itemize}
    \item Essential structure of variation in the input
    \item Way to soundly compile plain sub-trees to either a promised computation (Accumulation)
    \item Way to soundly compile plain sub-trees to a result (Evaluation)
    \item Way to select an alternative in the Variational Core
\end{itemize}
\end{minipage}}
\end{picture}
\end{frame}
\begin{frame}[label={sec:org2a36313}]{Variational Core: Add a new variant}
\begin{picture}(320,150)
\put(0,80){
   \begin{tikzpicture}
   \node (old) {\Tree [.$\vee$ [.$\vee$ [.$\chc[A]{}$ ] [. $s_{0}$ ] ]
                            [.$\wedge$ [.$\chc[D]{}$ ][.$\wedge$ [. $\chc[B]{}$ ][ .$s_{1}$ ] ] ] ]};
   \node (wedge) [right=0.25cm of old] {$\bigwedge$};
   \node (new) [right=0.25cm of wedge] {$c_{m+1.0} \wedge \textcolor{blue}{\chc[C]{c_{m+1.2},T}} \wedge \ldots \wedge c_{m+1.n}$}
   \end{tikzpicture}}
\end{picture}
\begin{block}{plain terms cached from previous variational core.}
\end{block}
\begin{block}{Thus will be looked up during evaluation/accumulation}
\end{block}
\end{frame}

\begin{frame}[label={sec:orgf0a60de}]{Variational Core: Add new variants}
\begin{picture}(320,150)
\put(0,10){
   \begin{tikzpicture}

   \Tree [.$\wedge$ [.$\vee$ [.$\vee$ [.$\chc[A]{}$ ] [. $s_{0}$ ] ]
                      [.$\wedge$ [.$\chc[D]{}$ ][.$\wedge$ [. $\chc[B]{}$ ][ .$s_{1}$ ] ] ] ]
         .NP \node(wh)[rectangle,fill=blue!20,draw,minimum height=20mm, yshift=-2em]{\textcolor{blue}{$VCore_{m+1}$}};]
   \end{tikzpicture}}
\end{picture}
\begin{block}{\phantom{m}}
\end{block}
\begin{block}{\phantom{m}}
\end{block}
\end{frame}

\section{Possible directions}
\label{sec:orgb37f0b5}

\begin{frame}[label={sec:orgda183af}]{Improve Performance}
\begin{block}{Migrate away from sbv}
\begin{block}{By implement a type system instead of using symbolic execution engine}
\end{block}

\begin{block}{Impact}
\begin{itemize}
\item Simpler more general implementation
\item Allow others to more easily implement a variational solver
\end{itemize}
\end{block}
\end{block}


\begin{block}{Improve sbv itself}
\begin{block}{How (already in progress)}
\begin{itemize}
\item Implement a benchmark library for sbv (Done)
\item Identify and reduce hot spots
\item The VSAT project has already contributed 4 PRs to sbv itself
\end{itemize}
\end{block}
\begin{block}{Lots of engineering work}
\end{block}
\end{block}
\end{frame}

\begin{frame}[label={sec:org3be4787}]{Extend Applications}
\begin{block}{How}
\begin{itemize}
\item Reuse paper survery from Jeff's Qualifier exam

\item Thomas' ideas + Feedback from SPLC community
\end{itemize}
\end{block}

\begin{block}{Benefits}
\begin{itemize}
\item Better Marketing
\item More Collaborators
\item More Papers
\item Increases impact of any future papers
\item Work is focused on papers rather than engineering or science
\end{itemize}
\end{block}


\begin{block}{Costs}
\begin{itemize}
\item Opportunity cost with respect to other directions
\item Could artificially restrict contribution
\end{itemize}
\end{block}
\end{frame}

\begin{frame}[label={sec:org49108d5}]{Abstract the Algorithm}
\begin{block}{Idea}
\begin{itemize}
\item Abstract the VSAT algorithm away from SAT solvers
\item Abstract the function calls in VSAT
\end{itemize}
\end{block}


\begin{block}{Benefits}
\begin{block}{Better impact than just a SAT solver}
\end{block}
\end{block}

\begin{block}{Costs}
\begin{block}{Opportunity cost with respect to other directions}
\end{block}
\end{block}
\end{frame}

\section{Technical Debt}
\label{sec:org07223f8}

\begin{block}{Tool needs a lot of cleanup}
\end{block}

\begin{block}{Soundness of the solver needs to be proved}
\end{block}
\end{document}