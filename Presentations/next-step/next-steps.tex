% Created 2020-05-31 Sun 16:35
% Intended LaTeX compiler: pdflatex
\documentclass[presentation]{beamer}
\usepackage[utf8]{inputenc}
\usepackage[T1]{fontenc}
\usepackage{graphicx}
\usepackage{grffile}
\usepackage{longtable}
\usepackage{wrapfig}
\usepackage{rotating}
\usepackage[normalem]{ulem}
\usepackage{amsmath}
\usepackage{textcomp}
\usepackage{amssymb}
\usepackage{capt-of}
\usepackage{hyperref}
\usepackage{minted}
\AtBeginSection[]{\begin{frame}<beamer>\frametitle{Topic}\tableofcontents[currentsection]\end{frame}}
\usepackage{tikz}
\usepackage{times}
\usepackage{amsmath}
\usepackage{verbatim}
\usetikzlibrary{shapes.geometric,backgrounds, positioning-plus,node-families,calc}
\usetheme{boxes}
\author{Jeffrey Young}
\date{2020-05-31}
\title{VSAT: What is Next}
\hypersetup{
 pdfauthor={Jeffrey Young},
 pdftitle={VSAT: What is Next},
 pdfkeywords={},
 pdfsubject={},
 pdfcreator={Emacs 26.3 (Org mode 9.3.6)},
 pdflang={English}}
\begin{document}

\maketitle
\begin{frame}{Outline}
\tableofcontents
\end{frame}


\section{Review: The Algorithm}
\label{sec:org8c3bc95}

\begin{frame}[label={sec:org4d2e1b5}]{The Algorithm}
\tikzstyle{na} = [baseline=-.5ex]
\tikzstyle{every picture}+=[remember picture]

\begin{itemize}[<+-| alert@+>]
    \item Coriolis acceleration
        \tikz[na] \node[coordinate] (n1) {};
\end{itemize}

% Below we mix an ordinary equation with TikZ nodes. Note that we have to
% adjust the baseline of the nodes to get proper alignment with the rest of
% the equation.
\begin{equation*}
\vec{a}_p = \vec{a}_o+\frac{{}^bd^2}{dt^2}\vec{r} +
        \tikz[baseline]{
            \node[fill=blue!20,anchor=base] (t1)
            {$ 2\vec{\omega}_{ib}\times\frac{{}^bd}{dt}\vec{r}$};
        } +
        \tikz[baseline]{
            \node[fill=red!20, ellipse] (t2)
            {$\vec{\alpha}_{ib}\times\vec{r}$};
        } +
        \tikz[baseline]{
            \node[fill=green!20,anchor=base] (t3)
            {$\vec{\omega}_{ib}\times(\vec{\omega}_{ib}\times\vec{r})$};
        }
\end{equation*}

\begin{itemize}[<+-| alert@+>]
    \item Transversal acceleration
        \tikz[na]\node [coordinate] (n2) {};
    \item Centripetal acceleration
        \tikz[na]\node [coordinate] (n3) {};
\end{itemize}

% Now it's time to draw some edges between the global nodes. Note that we
% have to apply the 'overlay' style.
\begin{tikzpicture}[overlay]
        \path[->]<1-> (n1) edge [bend left] (t1);
        \path[->]<2-> (n2) edge [bend right] (t2);
        \path[->]<3-> (n3) edge [out=0, in=-90] (t3);
\end{tikzpicture}
\end{frame}


\section{Possible directions}
\label{sec:orgc131194}

\subsection{Improve Performance}
\label{sec:org39f7dec}

\begin{frame}[label={sec:org251f047}]{Migrate away from sbv}
\begin{block}{How}
Implement a type system instead of using symbolic execution engine
\end{block}

\begin{block}{Impact}
\begin{itemize}
\item Simpler more general implementation
\item Allow others to more easily implement a variational solver
\end{itemize}
\end{block}
\end{frame}


\begin{frame}[label={sec:org5c73560}]{Improve sbv itself}
\begin{block}{How (already in progress)}
\begin{itemize}
\item Implement a benchmark library for sbv (Done)
\item Identify and reduce hot spots
\item The VSAT project has already contributed 4 PRs to sbv itself
\end{itemize}
\end{block}
\end{frame}


\begin{frame}[label={sec:org6c3dd08}]{Costs}
\begin{block}{Lots of engineering work}
\end{block}
\end{frame}

\subsection{Extend Applications}
\label{sec:org5f44031}

\begin{frame}[label={sec:org32f856c}]{Methods}
\begin{block}{Reuse paper survery from Jeff's Qualifier exam}
\end{block}

\begin{block}{Thomas' ideas + Feedback from SPLC community}
\end{block}
\end{frame}

\begin{frame}[label={sec:orge196c4f}]{Benefits}
\begin{block}{Better Marketing}
\end{block}

\begin{block}{More Collaborators}
\end{block}

\begin{block}{More Papers}
\end{block}

\begin{block}{Increases impact of any future papers}
\end{block}

\begin{block}{Work is focused on paper's rather than engineering or science}
\end{block}
\end{frame}

\begin{frame}[label={sec:org979dea1}]{Costs}
\begin{block}{Opportunity cost with respect to other directions}
\end{block}

\begin{block}{Could artificially restrict contribution}
\end{block}
\end{frame}

\subsection{Abstract the Algorithm}
\label{sec:org3697f00}

\begin{frame}[label={sec:orge5f0d29}]{Idea}
\begin{itemize}
\item Abstract the VSAT algorithm away from SAT solvers
\item Abstract the function calls in VSAT
\end{itemize}
\end{frame}


\begin{frame}[label={sec:org4df75fb}]{Benefits}
\begin{block}{Better impact than just a SAT solver}
\end{block}
\end{frame}

\begin{frame}[label={sec:org3b08328}]{Costs}
\begin{block}{Opportunity cost with respect to other directions}
\end{block}
\end{frame}

\section{Technical Debt}
\label{sec:org41dcd7c}

\subsection{Tool needs a lot of cleanup}
\label{sec:orgffee300}

\subsection{Soundness of the solver needs to be proved}
\label{sec:org08723af}
\end{document}